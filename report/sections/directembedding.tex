\section{Directembedding} % (fold)
\label{sec:Directembedding}
The architecture of the Directembedding library went through a major overhaul in this project.
The reification has been extended with new annotations and new capabilities such as language virtualization.
The reification is now highly customizable by the DSL author.
The library also aims to provide useful error messages where possible.

The following sections explain the improvements that have been made to the Directembedding library in this project.
For more details on how Directembedding works please consult the project's Github site, linked at the end of this report.
% \footnote{\href{https://github.com/directembedding/directembedding}{https://github.com/directembedding/directembedding}}.

\subsection{Architecture} % (fold)
\label{sub:Architecture}
Figure~\ref{fig:pipeline} shows the new architecture of Directembedding.
PreProcessing is an optional pass in the shallow embedding where the DSL author can transform the program in any way necessary before reification.
PreProcessing requires knowledge of the Scala reflection API.\
The DSLVirtualization pass performs the language virtualization explained in Section~\ref{sub:LanguageVirtualization}.
This pass happens in the shallow embedding.
\texttt{ReificationTransformation} is the major component of Directembedding and lifts the shallow embedding into the deep embedding.
In this pass, the metadata attached to the shallow embedding is used to reify the program into the DSL author's IR.\
\texttt{PostProcessing} is an optional pass through the deep embedding where the DSL author can transform the program in any way necessary before the program is passed back to the user.

% A valid DSL program must type-check in three stages of the pipeline: before PreProcessing, during DSLVirtualization and after PostProcessing.
% A well designed DSL should guarantee that a compilation error only occurs before PreProcessing or
% program compiles in the before

The entry point to using Directembedding is now \texttt{DETransformer}.
The design of the DETransformer is inspired by the \texttt{YYTransformer} in Yin-Yang~\autocite{jovanovic_yin-yang:_2014}, and uses \emph{mixin composition}~\autocite{odersky_scalable_2005} to compose a series of transformation stages.
An example Directembedding DSL is provided the \texttt{example} package object.

\begin{figure}
    \centering
    \begin{tikzpicture}
        [node distance=.5cm,
        start chain=going below,]
        \node[punktchain, join] (preprocess)  {PreProcessing};
        \node[punktchain, join] (virt)        {DSLVirtualization};
        \node[punktchain, join] (embed)       {ReificationTransformation};
        \node[punktchain, join] (postprocess) {PostProcessing};
    \end{tikzpicture}
    \caption{The Directembedding transformation pipeline.}\label{fig:pipeline}
\end{figure}

% subsection Architecture (end)
\subsection{Language virtualization} % (fold)
\label{sub:LanguageVirtualization}
Language virtualization is the process of converting standard language features into method calls, in order to give them arbitrary semantics.
Such language features include if-then-else statements, loops, and variable assignments.
It is not possible to override the semantics of such statements in Scala without macros.

Directembedding uses the language virtualization provided by the Yin-Yang~\autocite{jovanovic_yin-yang:_2014} framework.
This transformation happens in the \texttt{DSLVirtualization} pass.
The DSL author is able to configure which language features to override through the \texttt{DslConfig} trait.
The \texttt{LanguageVirtualizationSpec} shows 43 examples of how to use the language virtualization feature in Directembedding.
% subsection Language virtualization (end)

\subsection{Overriding predefined and third-party types} % (fold)
\label{sub:Overridingpredefinedandthirdpartytypes}
Directembedding supports the ability to override behavior of predefined and third-party types.
Predefined types are types provided by standard Scala libraries, such as \texttt{Int} and \texttt{String}.
Third-party types can be any types in a third-party library supported by the DSL.\

Reification for overriden types works the same way as reification with any other types.
The \texttt{typeMap} argument to \texttt{DETransformer} tells Directembedding where to to look for reification annotations.
If Directembedding does not find metadata to an invoked symbol, Directembedding will look for annotations on types in the \texttt{typeMap}.
This search on types and finding matching symbols is currently implemented in a naïve way, and could be improved in future implementations.
\texttt{TypeOverridingSpec} provides 6 examples of how to use the type overriding feature in Directembedding.
% subsection Overriding predefined and third-party types (end)

\subsection{Improved error messages} % (fold)
\label{sub:Improvederrormessages}
Much effort has been put into making error messages produced by Directembedding helpful.
These error messages broadly fall into two categories:
\begin{inparaenum}[i)]
    \item DSL author and
    \item end user error messages.
\end{inparaenum}


The error messages aimed at the DSL author are mostly meant to assist the author detect a DSL misconfiguration.
For instance, the configuration is now provided through a trait type parameter.
The trait determines the path from which all language virtualization and lift methods are implemented.
If the \texttt{compile} method---the receiver of DSL program after PostProcessing---is missing, Directembedding will fail with a compilation error indicating that the method is missing.
Another example is that if a reification annotation is used incorrectly, Directembedding will return a compilation error pointing to the misuse of the annotation.
Finally, detailed logging of all the steps of Directembedding transform can be enabled for debugging purposes.

The error messages aimed at the DSL user are mostly meant to surface incorrect use of the DSL in a user-friendly way.\
Most importantly, if the user invokes a method that is missing a reification annotation, Directembedding will return a compilation error saying that the method is not supported in the DSL, pointing to the culprit invocation in the DSL program.
Prior to this project, Directembedding threw a cryptic \texttt{EmptyIteratorException} in the same situation.
% subsection Improved error messages (end)

\subsection{Configurable reification} % (fold)
\label{sub:Configurablereification}
Many of the Directembedding features are now configurable by the DSL author.
The DSL author has increased control over how the reification is performed through new reification annotations beyond the original \texttt{@reifyAs} annotation.
Moreover, the DSL author has now fine-grained control over how literals are lifted into deep IR.\
For more details on the following configuration options, please refer to the Directembedding example DSLs and documentation.

\subsubsection{\texttt{reifyAs}} % (fold)
\label{sub:reifyAs}
The \texttt{@reifyAs} annotation is useful to move the invocation of a method in the shallow EDSL into any method in the deep EDSL.
This approach is inspired by related work on \emph{finally-tagless, polymorphic embedding}~\autocite{carette_finally_2009,hofer_polymorphic_2008}.
Listing~\ref{lst:reifyAs} shows an example usage of the \texttt{@reifyAs} annotation.
\lstinputlisting[label=lst:reifyAs, float, caption=\texttt{@reifyAs} example]{code/reifyAs.scala}
As seen in the example, the DSL author attaches the deep method as an argument to the \texttt{@reifyAs} annotation.
If the argument is missing, Directembedding will fail with a compilation error.

\subsubsection{\texttt{reifyAsInvoked}} % (fold)
\label{sub:reifyAsInvoked}
The \texttt{@reifyAsInvoked} annotation is useful to preserve the invocation a front-end on top an existing deep EDSL.\
Instead of invoking a static method as \texttt{@reifyAs}, the \texttt{@reifyAsInvoked} preserves the shallow embedding invocation order.
Listing~\ref{lst:reifyAsInvoked} shows an example usage of the \texttt{@reifyAsInvoked} annotation.
\lstinputlisting[label=lst:reifyAsInvoked, float, caption=\texttt{@reifyAsInvoked} example]{code/reifyAsInvoked.scala}
The deed query compiles only since the underlying \texttt{lifted.Query} has a matching \texttt{take} method.
The use of \texttt{@reifyAsInvoked} is therefore only encouraged in cases like slick-direct, where there is a close correspondance between the shallow EDSL and deep EDSL.
% subsection reifyAsInvoked (end)

\subsubsection{\texttt{passThrough}} % (fold)
\label{sub:passThrough}
The \texttt{@passThrough} annotation is useful to preserve values in the shallow DSL program.
By default, any invoked method in a DSL program should be reified during ReificationTransformation.
If a method is missing a reification annotation, the \texttt{ReificationTransformation} will return with a compilation error.\
Methods annotated with \texttt{@passThrough} skip reification in the ReificationTransformation.
Listing~\ref{lst:passThrough} shows an example usage of the \texttt{@passThrough} annotation.
\lstinputlisting[label=lst:passThrough, float, caption=\texttt{@passThrough} example]{code/passThrough.scala}
The example exhibits a subtle yet important difference between \texttt{@passThrough} and \texttt{@reifyAsInvoked}, the former will stop the reification at the invocation point and while the latter will recursively reify the arguments to the invoked method.
A careful observer may notice that the example may fail to compile, since the argument to the lifted query mustbe a lifted value.
For this reason, the use of \texttt{@passThrough} should only be used with great care.

% subsection passThrough (end)
\subsubsection{\texttt{customLifts}} % (fold)
\label{sub:customLifts}
% Can't we solve this with implicit conversions? E.g., lift[T, U](t: T)(c: Converter[T, U]): U
By default, all literal are lifted through a method with the following signature
\lstinputlisting{code/lift.scala}
where \texttt{Rep} is the supertype of all elements in the IR.\
The issue with default configuration is that it ignores the hierarchy of the IR, all literals will have the type \texttt{Rep} although they may be lifted into a subtype of \texttt{Rep}.
The \texttt{customLifts} parameter to \texttt{DETransformer} alleviates this issue by giving the DSL author fine grained control over which types are lifted into which IR types.
For instance, the DSL author can provide a custom lift for values of type \texttt{Int} and another lift method for values of type \texttt{String}.
Directembedding does not enforce that the return type of a custom lift methods is a subtype of \texttt{Rep}

% subsection customLifts (end)
% same here
\subsubsection{\texttt{liftIgnore}} % (fold)
\label{sub:liftIgnore}
The \texttt{liftIgnore} configuration parameter allows the DSL author to list which literals should not be lifted during ReificationTransformation.
This can be useful if certain literals are introduced in the PreProcessing step which should not be lifted.
% subsection liftIgnore (end)

% subsection C (end)


% section Directembedding (end)
